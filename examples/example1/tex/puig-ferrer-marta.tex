\documentclass[11pt,catalan]{article}

\usepackage[ca,NIC]{examen}

\data{11 de novembre de 2020}
\examen{Primer Parcial}
\quadrimestre{Tardor}
\curs{2020-2021}
\assignatura{\`Algebra Lineal}
\cognoms{Puig Ferrer}
\nom{Marta}

\begin{document}

\begin{enunciat}
Trobeu la inversa de la matriu 
\[
  A = \begin{pmatrix}{*{3}r} 0 & -1 & 3\\ 2 & 1 & -3\\ -1 & 0 & 3\end{pmatrix}\,.
\]
\end{enunciat}

\begin{quadricula}
\begin{tabular}{|L{5cm}{1.25cm}|}
\hline
$A^{-1}=$ \\
\hline
\end{tabular}
\end{quadricula}

\begin{solucio}
\begin{center}
\begin{tabular}{|C{5cm}{1.25cm}|}
\hline
$A^{-1}=\deufrac{1}{6}\begin{pmatrix}{*{3}r} 3 & 3 & 0\\ -3 & 3 & 6\\ 1 & 1 & 2\end{pmatrix}$ \\
\hline
\end{tabular}
\end{center}
\end{solucio}


\begin{enunciat}
Es considera el pla vectorial $P$ de $\rn 3$
\[
  P = \langle (2,1,-2), (1,1,-2) \rangle
\]
i la base $\cal B'=\{(-2,1,-1),(2,-1,2),(1,-1,1)\}$. Determineu l'equació implícita del pla $P$ respecte de la base $\cal B'$.
\end{enunciat}

\begin{quadricula}
\begin{tabular}{|C{5cm}{0.5cm}|}
\hline
 \\
\hline
\end{tabular}
\end{quadricula}

\begin{solucio}
\begin{center}
\begin{tabular}{|C{5cm}{0.5cm}|}
\hline
$x' - z' = 0$ \\
\hline
\end{tabular}
\end{center}
\end{solucio}


\begin{enunciat}
Resoleu el següent sistema d'equacions
\[
  \left.\aligned - x + y - z &= 0 \\ 3 x - 2 y + 2 z &= -3 \\ x - y + 2 z &= 2 \\ - x + 2 y - 2 z &= -3 \endaligned\;\right\}\,,
\]
\end{enunciat}

\begin{quadricula}
\begin{tabular}{|C{5cm}{0.5cm}|}
\hline
  \\
\hline
\end{tabular}
\end{quadricula}

\begin{solucio}
\begin{center}
\begin{tabular}{|C{5cm}{0.5cm}|}
\hline
$x=-3$, $y=-1$, $z=2$. \\
\hline
\end{tabular}
\end{center}
\end{solucio}


\begin{enunciat}
Determineu l'equació implícita del pla perpendicular a la recta 
\[
  x - 1 = y - 3 = \frac{z - 2}{2}
\]
que passa pel punt $(1,1,-1)$.
\end{enunciat}

\begin{quadricula}
\begin{tabular}{|C{5cm}{0.5cm}|}
\hline
  \\
\hline
\end{tabular}
\end{quadricula}

\begin{solucio}
\begin{center}
\begin{tabular}{|C{5cm}{0.5cm}|}
\hline
$x + y + 2 z = 0$ \\
\hline
\end{tabular}
\end{center}
\end{solucio}


\end{document}